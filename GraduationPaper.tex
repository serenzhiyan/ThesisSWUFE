%!TEX program = xelatex
% 完整编译: xelatex -> bibtex -> xelatex -> xelatex
\documentclass[lang=cn,11pt,a4paper]{elegantpaper}

\title{基于删失数据的非参数库存管理算法}
\author{Ethan DENG \\ Fudan University \and Dongsheng DENG \\ PA Technology}
\institute{\href{https://elegantlatex.org/}{Elegant\LaTeX{} 项目组}}

\version{0.08}
\date{\zhtoday}

\begin{document}

\maketitle

\begin{abstract}
本文为 \href{https://github.com/ElegantLaTeX/ElegantPaper/}{ElegantPaper} 的说明文档。此模板基于 \LaTeX{} 的 article 类,专为工作论文写作而设计。设计这个模板的初衷是让作者不用关心工作论文的格式,专心写作,从而有更加舒心的写作体验。如果你有其他问题、建议或者报告 bug,可以在 \href{https://github.com/ElegantLaTeX/ElegantPaper/issues}{Github::ElegantPaper/issues} 留言。如果你想了解更多 Elegant\LaTeX{} 项目组设计的模板,请访问 \href{https://github.com/ElegantLaTeX/}{Github::ElegantLaTeX}。
\keywords{非参数算法;删失数据;}
\end{abstract}



\section{研究背景及意义}

\subsection{研究背景}
库存管理是指在物流过程和存储销售过程中商品数量的管理,现代管理学认为零库存是最好的库存管理。库存多,占用资金多,利息负担加重。但是如果过分降低库存,则会出现断档,不能满足日常的需求。库存管理需要根据外界对库存的要求、企业订购的成本,进行预测,计划和执行补充库存,并对这种行为进行控制,重点在于确定如何订货,订购多少,何时订货。

从运筹学的早期开始,库存控制和计划问题就受到了业界和学术界的浓厚兴趣。该领域的早期文献将需求建模为确定性的并具有已知数量,但是很快就变得很明显,确定性建模通常不足,在建模未来需求时需要纳入不确定性。

引入不确定即需要将需求设为随机变量,该随机变量可能服从某种特定的分布,依据需求变量取值的概率进行订购决策。事先假设需求变量服从的分布,使用这种参数化的方法会得到性质优良的订购方法,较低的订购成本,但是问题在于实际情况中,我们往往不清楚或者不能获得关于分布的完全信息,这也是参数方法在实际应用中的阻碍,因此非参数库存管理方法的研究就显得比较重要了。事实上,非参数方法是研究库存计划里非常重要的一类方法,具有较高的应用价值,与参数化的方法生成的订购决策相比,订购成本就相对来说会高一点,故希望设计出来的非参数方法有比较快的收敛速度,使每期的订购量在尽量短的时间内收敛到参数化方法(比如经典库存理论中的报童模型)给出的最优订购量,以减少销售损失和订购成本,同时希望T期内的订购量的方差尽可能小一点,如果订购量的变化起伏比较大的话,不利于库存管理。

\section{国内外研究现状和发展趋势}
\subsection{研究方法综述}

库存管理的研究方法大致分为以下几类:经典库存理论、贝叶斯方法、非参数方法,以及比较新的方法——在线凸优化。
\subsubsection{经典库存理论}
尽管库存管理人员不知道未来实现的需求,但在做出库存订购决策时,可以假设完全知道其分布的信息。最著名的随机库存问题是报童模型,其目标是将单个时期内预期的过量成本和缺量成本最小化。该问题的最佳解决方案对应于潜在的需求分布的临界值点(即关于过量成本和缺量成本的一个比率)。无论多余的库存是否容易腐烂,基于报童模型的基本库存策略都是最佳的。
报童模型中的成本表达式:
\begin{equation}
    Cost\left( Q,D\right) =
    {\begin{cases}
        C_{o}\left( Q-D\right) , & \text{if}\,D<Q \\
        C_{u}\left( D-Q\right) , & \text{if}\,D\geq Q.
    \end{cases}}
\end{equation}
因为D是一个随机变量,所以当计算成本时,应该使用期望值:

当期望成本最小时,即采用对Q求导的方式寻找最佳订货量的时候:
 
因此得到了最佳订货量应该满足的条件:

\subsubsection{贝叶斯方法}

当无法获得有关需求分布的信息时,文献中最常见的方法是使用贝叶斯更新。在这种方法下,库存管理人员只能获得有限的需求信息。

一般情况假设知道需求所属的分布族,但是不确定其参数。对参数值的不确定性具有初步的先验信念,并且通过计算后验分布根据历史实现的需求不断更新此信念。1960年Scarf,Karlin和Iglehart等早期论文考虑了需求分布属于指数和范围族的情况。Lovejoy在1990年发表的文章表明,基于临界分位数的简单渐近库存策略是最优的或接近最优的。在以上方法中,知道历史已实现的需求,不论是低于还是高于库存水平。1976年Conrad指出了需求量与销售量之间的对比,他指出了删失在估计泊松需求分布参数中的作用。在考虑了销售损失不可观察的贝叶斯文献中,Harpaz等人与1982年发现在易腐库存的情况下,最佳库存量高于渐近的解决方案,直观地理解是:因为通过增加库存,更有可能获得更准确,未经删失的需求信息。Lariviere和Porteus(1999)使用称为“报童分布”的特殊分布(Braden和Freimer)为持有较高库存的结果提供了足够的条件。当前关于不可观察到的销售损失和删失需求的所有文献主要集中在贝叶斯框架上,在贝叶斯框架中,需求的后验分布是根据观察到的销售数据进行更新的。在贝叶斯方法中,有文献指出有时很难更新先验分布。另外,在许多应用中,不清楚应该使用哪个特定的先验分布。
\subsubsection{非参数方法}

在非参数方法中,往往需求分布未知,需求所属的分布族未知,仅根据历史数据,检查平均预期成本收敛到某基准的速率。Burnetas和Smith于2000年使用基于随机逼近算法的一种变体,开发了一种适用于库存易腐时的定购和定价的自适应算法,使用删失的需求样本来找到需求分布的临界分位数。他们表明平均利润收敛到最优值,但没有建立收敛速度。凹面自适应值估计(CAVE)算法是利用删失数据通过识别报童模型成本函数的凸度来评估报童模型成本函数的非参数方法。该算法使用一系列分段线性函数逐次逼近成本函数。当库存容易腐烂时,Godfrey和Powell(2001)表明CAVE算法具有良好的数值性能,但没有证明任何收敛结果。以上方法仅适用于易腐库存情况。面对未经删失的样本,当管理人员对需求分布的信息受限(例如均值和标准差)时,可以采用Moon1993年提出的方法,目标是计算最佳库存量,该最佳库存量将提供最大的预期利润应对不理想的需求。同样当需求分布中有未经删失的样本可用时,2007年Levi等研究多周期库存系统,而无需了解需求分布。他们以高概率计算达到一定准确度所需的样本量。

\subsubsection{在线凸优化}

与常规凸面优化一样,在线凸面优化的目的是最小化在凸紧集上定义的凸函数,优化器在算法开始时并不知道目标函数,并且在每次迭代时,都会根据到目前为止可获得的信息选择可行的解决方案,承担与其决策相关的开销,并获得有关该问题的一些相关信息,有研究表明,当该信息为当前解中目标函数的梯度时,平均T周期成本以一定的比率收敛至最优成本。

\subsection{主要研究内容}
主要研究内容就是设计一个改进自适应库存管理AIM算法的非参数的方法。首先对于AIM算法,它的基本假设是库存管理人员没有关于未来需求分布的先验信息,而仅观察销售数据,也就是说我们得到的是删失的需求。其创新的地方在于:基于在线凸优化,利用成本函数的凸性,并在每次迭代中利用梯度下降,有一定的收敛速度收敛到报童模型的基准。该算法适用于易腐和不易腐烂的库存。此外,证明了两种情况下算法的收敛速度。如果库存不易腐烂,则每个阶段的决策受早期阶段中决策的约束。在这种情况下,通过在单个服务窗口的GI/D/1队列中等待时间过程与随机梯度下降方法之间建立联系来引入一种新的证明方法,其中服务时间的参数与随机梯度下降法的步长有关,证明了算法的收敛速度。

在AIM的基本假设(需求分布未知,并且只能获得历史删失需求)的基础上,希望设计一个算法可以利用更多的历史信息,因为AIM只使用了上一期的销售数据来预测下一期的订购量,因此如果能够使用更多的历史信息,那么直观上能够达到更好的收敛速度,更低的订购成本和T期订购量的方差,这是研究的出发点。设计算法之后希望能从数值验证和理论推导的两个角度给出算法的收敛速度。



\textbf{注意}:Elegant\LaTeX{} 系列模板已经全部上传至 \href{https://www.overleaf.com/latex/templates/elegantpaper-template/yzghrqjhmmmr}{Overleaf} 上,用户可以在线使用。另外,为了方便国内用户,模板也已经传至\href{https://gitee.com/ElegantLaTeX/ElegantPaper}{码云}。


\subsection{全局选项}
此模板定义了一个语言选项 \lstinline{lang},可以选择英文模式 \lstinline{lang=en}(默认)或者中文模式 \lstinline{lang=cn}。当选择中文模式时,图表的标题引导词以及参考文献,定理引导词等信息会变成中文。你可以通过下面两种方式来选择语言模式:
\begin{lstlisting}
\documentclass[lang=cn]{elegantpaper} % or
\documentclass{cn}{elegantpaper} 
\end{lstlisting}

\textbf{注意:} 英文模式下,由于没有添加中文宏包,无法输入中文。如果需要输入中文,可以通过在导言区引入中文宏包 \lstinline{ctex} 或者加入 \lstinline{xeCJK} 宏包后自行设置字体。 
\begin{lstlisting}
\usepackage[UTF8,scheme=plain]{ctex}
\end{lstlisting}

\subsection{英文与数学字体}

本模板使用 \lstinline{newtxtext} 和 \lstinline{newtxmath} 分别设置全文的英文文本字体和数学字体。数学字体的效果如下:
\begin{equation}
(a+3b)^{n} = \sum_{k=0}^{n} C_{n}^{k} a^{n-k} (3b)^k\label{eq:binom}
\end{equation}

\subsection{自定义命令}
此模板并没有修改任何默认的 \LaTeX{} 命令或者环境\footnote{目的是保证代码的可复用性,请用户关注内容,不要太在意格式,这才是本工作论文模板的意义。}。另外,我自定义了 4 个命令:
\begin{enumerate}
  \item \lstinline{\email}:创建邮箱地址的链接,比如 \email{ddswhu@outlook.com};
  \item \lstinline{\figref}:用法和 \lstinline{\ref} 类似,但是会在插图的标题前添加 <\textbf{图 n}> ;
  \item \lstinline{\tabref}:用法和 \lstinline{\ref} 类似,但是会在表格的标题前添加 <\textbf{表 n}>;
  \item \lstinline{\keywords}:为摘要环境添加关键词。
\end{enumerate}

\subsection{参考文献}
此模板使用 \hologo{BibTeX} 来生成参考文献,中文模式下默认使用的文献样式(bib style)是 \lstinline{GB/T 7714-2015}\footnote{通过调用 \href{https://ctan.org/pkg/gbt7714}{\lstinline{gbt7714}} 宏包}。参考文献示例:~\cite{en3} 使用了中国一个大型的 P2P 平台(人人贷)的数据来检验男性投资者和女性投资者在投资表现上是否有显著差异。

你可以在谷歌学术,Mendeley,Endnote 中获得文献条目(bib item),然后把它们添加到 \lstinline{wpref.bib} 中。在文中引用的时候,引用它们的键值(bib key)即可。注意需要在编译的过程中添加 \hologo{BibTeX} 编译。

本模板还添加了 \lstinline{cite=numbers} 、\lstinline{cite=super} 和 \lstinline{cite=authoryear}  三个参考文献选项,用于设置参考文献格式的设置,默认为 \lstinline{numbers}。理工科类一般使用数字形式 \lstinline{numbers} 或者上标形式 \lstinline{super},而文科类多使用作者-年份 \lstinline{authoryear} 比较多。如果需要改为 \lstinline{cite=numbers}  或者  \lstinline{authoryear} ,可以使用
\begin{lstlisting}
\documentclass[cite=super]{elegantpaper} % super style ref style
\documentclass[super]{elegantpaper}

\documentclass[cite=authoryear]{elegantpaper} % author-year ref style
\documentclass[authoryear]{elegantpaper}
\end{lstlisting}


\section{协作人员招募}
招募 Elegant\LaTeX{} 的协作人员,没有工资。工作内容:翻译 Elegant\LaTeX{} 系列模板相关的文稿(中翻英),维护模板的 wiki(主要涉及 Markdown),如果有公众号文稿写作经历的话,也可以帮忙写微信稿。本公告长期有效。

目前 ElegantLaTeX 共有 4 名协作人员,分别是
\begin{itemize}
  \item 官方文档翻译: \href{https://github.com/peggy2006xzyz}{YPY};
  \item Github 维基维护: \href{https://github.com/izinngo}{Ingo Zinngo}、\href{https://github.com/xiaohao890809}{追寻原风景};
  \item QQ 群管理员: \href{https://github.com/sikouhjw}{Sikouhjw}.
\end{itemize}

在此感谢他们无私的奉献!


\section{致谢}
截止到 2019 年 10 月 17 日,ElegantPaper v0.08 版本发布,ElegantPaper 模板在 Github 上的收藏数(star)达到了 164。在此特别感谢 China\TeX{} 以及 \href{http://www.latexstudio.net/}{\LaTeX{} 工作室}对于本系列模板的大力宣传与推广。

如果你喜欢我们的模板,你可以在 Github 上收藏(Star)我们的模板。
\begin{figure}[htbp]
  \centering
  \includegraphics[width=\textwidth]{star.png}
  \caption{一键三连求赞}
\end{figure}

\section{捐赠}
如果您非常喜爱我们的模板,你还可以选择捐赠以表达您对我们模板和我的支持!

\begin{figure}[htbp]
  \centering
  \includegraphics[width=0.5\textwidth]{donate.jpg}
\end{figure}

\textbf{赞赏费用的使用解释权归 Elegant\LaTeX{} 所有,并且不接受监督,请自愿理性打赏}。10 元以上的赞赏,我们将列入捐赠榜,谢谢各位金主!

\begin{table}[!htbp]
  \centering
  \caption{Elegant\LaTeX{} 系列模板捐赠榜}
  \begin{tabular}{crcc}
    \toprule
    捐赠者   & 金额 & 时间 & 渠道 \\
    \midrule
    Lerh  & 10 元  & 2019/05/15 & 微信 \\
    越过地平线 & 10 元    & 2019/05/15 & 微信 \\
    大熊 &  20 元 & 2019/05/27 & 微信 \\
    * 空 & 10 元 & 2019/05/30 & 微信\\
    \href{http://www.latexstudio.net/}{latexstudio.net} & 666 元 & 2019/06/05 & 支付宝\\
    Cassis & 11 元 & 2019/06/30 & 微信\\
    * 君 & 10 元 & 2019/07/23 & 微信\\
    * 萌 & 19 元 & 2019/08/28 & 微信 \\
    曲豆豆 & 10 元 & 2019/08/28 & 微信 \\
    李博 & 100 元 & 2019/10/06 & 微信\\
    Njustsll & 10 元 & 2019/10/11 & 微信 \\
  \bottomrule
  \end{tabular}%
\end{table}%

\section{常见问题 FAQ}

\begin{enumerate}[label=\arabic*).]
  \item \textit{如何删除版本信息?}\\
      导言区不写 \lstinline|\version{x.xx}| 即可。
  \item \textit{如何删除日期?}\\
      需要注意的是,与版本 \lstinline{\version} 不同的是,导言区不写或注释 \lstinline{\date} 的话,仍然会打印出当日日期,原因是 \lstinline{\date} 有默认参数。如果不需要日期的话,日期可以留空即可,也即 \lstinline|\date{}|。
  \item \textit{如何获得中文日期?}\\
      为了获得中文日期,必须在中文模式下\footnote{英文模式下,由于未加载中文宏包,无法输入中文。},使用 \lstinline|\date{\zhdate{2019/10/11}}|,如果需要当天的汉化日期,可以使用 \lstinline|\date{\zhtoday}|,这两个命令都来源于 \href{https://ctan.org/pkg/zhnumber}{\lstinline{zhnumber}} 宏包。
  \item \textit{如何添加多个作者?}\\
      在 \lstinline{\author} 里面使用 \lstinline{\and},作者单位可以用 \lstinline{\\} 换行。\begin{lstlisting}
\author{author 1\\ org. 1 \and author 2 \\ org. 2 }
\end{lstlisting}
  \item \textit{如何添加中英文摘要?}\\
      请参考 \href{https://github.com/ElegantLaTeX/ElegantPaper/issues/5}{Github::ElegantPaper/issues/5}
\end{enumerate}

\section{示例}

为了让大家更加清楚最终的论文效果,如下给出两篇使用 ElegantPaper 模板排版的工作论文示例,也欢迎大家“投稿”!

\begin{enumerate}
  \item \href{https://github.com/EthanDeng/bank-custody}{银行存管、投资者决策与 P2P 网络借贷规范发展};
  \item \href{https://github.com/EthanDeng/risk-awareness}{互联网金融风险与投资者风险意识 —— 来自网贷平台交易数据的证据}。
\end{enumerate}

这是一个最小示例文档(Minimal Example):
\begin{lstlisting}
\documentclass[lang=cn,a4paper,11pt]{elegantpaper}

% title information
\title{Working Paper Example}
\author{Author Name} 
\institute{Elegant\LaTeX{} Group}

\version{1.00}
\date{\today}

\begin{document}

\maketitle

\begin{abstract}
Your abstract goes here.
\keywords{keyword1, keyword2}
\end{abstract}

\section{Introduction}
The content of introduction section.

\section{Conclusion}
The content of conclusion section.

\bibliography{wpref}

\end{document}
\end{lstlisting}

\nocite{*}
\bibliography{wpref}

\end{document}
